%        File: pyne_ans.tex 
%     Created: 6.28.2012 

\documentclass{anstrans}
%%%%%%%%%%%%%%%%%%%%%%%%%%%%%%%%%%%
\title{PyNE : Python For Nuclear Engineering}
\author{Anthony~M.~Scopatz$^1$, Paul~K.~Romano$^2$, Paul~P.H.~Wilson$^3$, Kathryn~D.~Huff$^3$}

%% uncomment these next five only if using anstrans
\institute{$^1$ University of Chicago, $^2$ Massachusetts Institute of Technology, $^3$ University of Wisconsin }
\email{pyne-dev@googlegroups.com}
\usepackage{graphicx}
\usepackage{booktabs} % nice rules for tables
\usepackage{microtype} % if using PDF
\newcommand{\units}[1] {\:\text{#1}}%
\newcommand{\SN}{S$_N$}%{S$_\text{N}$}%{$S_N$}%

\date{}
%%%%%%%%%%%%%%%%%%%%%%%%%%%%%%%%%%%
\begin{document}
%%%%%%%%%%%%%%%%%%%%%%%%%%%%%%%%%%%%%%%%%%%%%%%%%%%%%%%%%%%%%%%%%%%%%%%%%%%%%%%%
\section{Introduction}

PyNE, or `Python for Nuclear Engineering'\footnote{http://pyne.github.com}, is a
nascent free and open source C++/Cython/Python package for performing common
nuclear engineering tasks.  This is intended as a base level tool kit - akin to
SciPy or Biopython - for common algorithms in the nuclear science and
engineering domain.

\section{Background}

While PyNE may be considered long overdue from an external perspective, the 
nuclear industry poses a uniquely high barrier to entry for free software.  
That closed source solutions are often easier to develop is an irony that 
increases the novelty of this package.

An initial hurdle for PyNE is the myriad of arcane file formats (both plain text
and binary) which have been industry standard since the 1960s.  Though obtuse
specifications are not solely a nuclear problem, the scale of the number of
formats is much greater.  Partly to blame is that some of these formats were
enshrined in international law at a time when Fortran 2 was the language of
choice.

Thus the emerging value of PyNE is that it allows new users to shortcut the
tedious and error prone process of writing their own tools to parse these data
and output file formats.  The current unfortunate state of affairs is due in
part to huge institutional inertia on the part of the maintainers of such
formats.  This manifests as a reluctance to develop or refactor new and existing
codes in a modern open source manner.  PyNE seeks to increase human efficiency
via a shared set of solutions rather than having every developer around the
world replicate the same parsing routines.

Another major challenge for the community of PyNE developers is maintaining the
BSD license while explicitly avoiding any code which may be subject to export
control.  Many nuclear engineering codes are open source in the sense that the
source code is distributed to developers who are then free to modify it.
However, due to the fear of illicit use for the development of weapons, these
same codes are then heavily export controlled and redistribution in any form is
against their licenses and is often illegal.

Redistribution concerns are not limited to source code.  Basic nuclear data,
while fundamentally un-copywritable under Western jurisprudence, may also be
considered sensitive and subject to export control laws.  PyNE has developed a
three-tiered strategy to alleviate the data burden of the individual user based
on the level of openness of the data.

In spite of the above administrative concerns, PyNE's place in the nuclear
software ecosystem requires that it have a general architecture.  Large portions
of the code base are written in pure C or C++ and are built as
Python-independent shared libraries.  This enables other, compiled nuclear
engineering codes to leverage PyNE.  Hence, the Cython layer has wrappers for
C++ standard library containers (maps, sets, lists, etc) which implement the
Python collections interface of the appropriate type.  Because of the high
degree of factorization in PyNE, these wrappers could easily be reused by other
projects.

%%%%%%%%%%%%%%%%%%%%%%%%%%%%%%%%%%%%%%%%%%%%%%%%%%%%%%%%%%%%%%%%%%%%%%%%%%%%%%%%

\section{Current Capabilities}

The following is an overview of the current capabilities of PyNE.  This covers
briefly each major module and its relevance to nuclear engineering.  This is not
meant to be a tutorial.  For such information please refer to the PyNE user's
guide \cite{PyNE:2012}.

\subsection{Nuclide Naming} 
There are a plethora of ways to represent nuclide names.  The
\texttt{pyne.nucname} module may be used to convert between these various naming
schemes. Currently the following naming conventions are supported: zzaaam, human
readable names, MCNP, Serpent, NIST, and CINDER.  This module may convert
between any of them.  Furthermore, it is implemented in C.


\subsection{Basic Nuclear Data}
The \texttt{pyne.data} module aims to provide quick access to very high fidelity
nuclear data. Usually values are taken from the \texttt{nuc\_data.h5} library.
This library is generated by the new \texttt{nuc\_data\_make} utility at install
time.  Current data includes atomic masses, decay data, neutron scattering
lengths, and simple cross section data. 63-group neutron cross sections, photon
cross sections, and fission product yields are also added when CINDER is
available on the machine of the user.  This module is implemented in C++.


\subsection{The Material Class} 
Materials are the primary container for radionuclides throughout PyNE. They map
nuclides to mass weights, though they also contain methods for converting to and
from atom fractions.  In many ways this class takes inspiration from numpy
arrays and python dictionaries.  Materials are implemented in C++ and support
both text and HDF5 persistence.  This class may be found within
\texttt{pyne.material}.


\subsection{CCCC Formats}
The \texttt{pyne.cccc} module contains a number of classes for reading various
cross section, flux, geometry, and data files with specifications given by the
Committee for Computer Code Coordination. The following types of files can be
read using classes from this module: ISOTXS, DLAYXS and SPECTR, with plans to
support BRKOXS, RTFLUX, ATFLUX, RZFLUX, and MATXS.

\subsection{ACE Format Cross Sections}

One of the most common data formats used to represent pointwise
linearly-interpolatable cross sections is the ACE (A Compact ENDF) format. This
format was introduced by LANL for use in the MCNP \cite{mcnp} code and is now
used by the Serpent \cite{serpent} and OpenMC \cite{openmc} Monte Carlo codes as
well. While there are a variety of resources that allow a user to inspect and
plot data directly from ENDF, there has generally been no good means of looking
at processed data in the ACE format that is actually used in Monte Carlo
simulations. As such, a module has been added to PyNE to parse and store data
from ASCII or binary ACE files. A front-end graphical user interface is also in
development that will give the user an easy interface to view and compare both
cross sections and secondary energy and angle
distributions. Figure~\ref{fig:ace-gui} shows cross section data for Pu-239
being plotted within the GUI.
\begin{figure}[ht]
  \centering
  \includegraphics[width=3.3in]{ace-gui.png}
  \caption{Front-end graphical user interface for plotting ACE format data in
    PyNE.}
  \label{fig:ace-gui}
\end{figure}


\subsection{Cross Section Interface} 
The \texttt{pyne.xs} sub-package provides a top-level interface for computing
(and caching) multigroup neutron cross sections. These cross sections will be
computed from a variety of available data sources.  Nominally these are stored
in the \texttt{nuc\_data.h5} HDF5 library which was generated by
\texttt{nuc\_data\_make} at installation time.  The library is searched for the
following data sets in the following order of preference:

\begin{enumerate}
    \item CINDER 63-group cross sections \cite{cinder},
    \item A two-point fast/thermal interpolation (using `simple\_xs' data from
      KAERI \cite{kaeri}),
    \item or physical models implemented in this sub-package.
\end{enumerate}

In the future, this package will support generating multigroup cross sections
from user-specified pointwise data sources (such as ENDF or ACE files).  In
future versions of PyNE, this package may also include SCALE multigroup cross
sections, if available \cite{scale}.


\subsection{ORIGEN 2.2 Support}
The \texttt{pyne.origen22} module provides an interface for reading, writing,
and merging certain ORIGEN 2.2 \cite{origen} input and output files.
Specifically, tapes 4, 5, 6, and 9 are currently supported.  Other decks could
be supported in the future with relative ease if the need or interest arises.


\subsection{Serpent Support} 
Serpent \cite{serpent} is a continuous energy Monte Carlo reactor physics code.
Pyne contains support for reading in Serpent's three types of output files:
\texttt{*.res}, \texttt{*.dep}, and \texttt{*.det}.  These are all in Matlab
\texttt{*.m} format and are read in as Python dictionaries of numpy arrays and
Materials.  They may be optionally written out to corresponding \texttt{*.py}
files and imported later.

\subsection{MCNP5 Support}
MCNP5 \cite{mcnp} is a general purpose continuous energy Monte Carlo radiation
transport code.  Pyne contains support for reading some data from MCTAL files
and for reading and writing surface source files (SSW/SSR).  Additions are
underway for reading all MCTAL data and mesh tally (MESHTAL) data.

\subsection{NJOY Module}
Recently the developers of PyNjoy \cite{dragon}, a set of Python bindings for
processing nuclear data using NJOY, agreed to have their work integrated with
the PyNE project as a new NJOY module. This will help to encourage active
development and wider participation in the project.

The \texttt{pyne.njoy} module provides a simple method of generating nuclear
data in a variety of formats from raw ENDF data without the user having to worry
about the exact format of the NJOY input files. Thus, with a single Python
script, it is possible to process thousands of nuclides at once rather than
managing many NJOY input files.


%%%%%%%%%%%%%%%%%%%%%%%%%%%%%%%%%%%%%%%%%%%%%%%%%%%%%%%%%%%%%%%%%%%%%%%%%%%%%%%%
\section{Conclusions}
Over the past year, the PyNE development team has formed a community which has
successfully overcome many initial hurdles.  In addition to the regular burden
of starting a new open source project, the nuclear domain contains special
concerns about export control and non-proliferation.  By maintaining a
conservative development strategy, only methods and data which are truly open
make it into the PyNE code base.  Even with these restraints the number of tools
possible is very large.

With over 22000 lines of C++, Cython, and Python code, PyNE has just begun to
implement what is in its purview.  The more modules that PyNE has, the more
utilities may be composed using it.  Noting the possibility for `scope
explosion,' the development team is committed to adhering to modern software
quality development standards.  At a minimum, all contributions to PyNE must be
tested, documented, and follow the official Python coding standard.

Future work may include further NJOY-like capabilities, a fully-featured MCNP
interface, a common nuclear materials database, various infinite medium
diffusion solvers, deep geologic repository modeling functionality, and Bateman
equation solvers.

%%%%%\acknowledgements
\section*{Acknowledgements}
The authors would like to recognize additional code contributions by Christopher
Dembia, Robert Flanagan, and Eric Relson.  Moreover, we would like to thank Seth
Johnson, Joshua Peterson, Rachel Slaybaugh, Nick Touran, and Morgan White for
their inspiration, guidance, and testing.


%%%%%%%%%%%%%%%%%%%%%%%%%%%%%%%%%%%%%%%%%%%%%%%%%%%%%%%%%%%%%%%%%%%%%%%%%%%%%%%%
\nocite{*}
\bibliographystyle{ans}
\bibliography{pyne_ans2012}
\end{document}


